%%%%%%%%%%%%%%%%%%%%%%%%%%%%%%%%%%%%%%%%%
% Short Sectioned Assignment
% LaTeX Template
% Version 1.0 (5/5/12)
%
% This template has been downloaded from:
% http://www.LaTeXTemplates.com
%
% Original author:
% Frits Wenneker (http://www.howtotex.com)
%
% License:
% CC BY-NC-SA 3.0 (http://creativecommons.org/licenses/by-nc-sa/3.0/)
%
%%%%%%%%%%%%%%%%%%%%%%%%%%%%%%%%%%%%%%%%%
\documentclass[paper=a4, fontsize=11pt]{scrartcl} % A4 paper and 11pt font size
\usepackage[T1]{fontenc}
\usepackage{fourier}
\usepackage[english]{babel}
\usepackage{amsmath,amsfonts,amsthm}

\usepackage{lipsum}

\usepackage{sectsty}
\allsectionsfont{\centering \normalfont\scshape}

\usepackage{fancyhdr}
\pagestyle{fancyplain}
\fancyhead{}
\fancyfoot[L]{} % Empty left footer
\fancyfoot[C]{} % Empty center footer
\fancyfoot[R]{\thepage} % Page numbering for right footer
\renewcommand{\headrulewidth}{0pt} % Remove header underlines
\renewcommand{\footrulewidth}{0pt} % Remove footer underlines
\setlength{\headheight}{13.6pt} % Customize the height of the header

\numberwithin{equation}{section}
\numberwithin{figure}{section}
\numberwithin{table}{section}

\setlength\parindent{0pt}

\usepackage{graphicx}
\graphicspath{ {./images/} }

\usepackage{listings}
\usepackage{color}

\definecolor{dkgreen}{rgb}{0,0.6,0}
\definecolor{gray}{rgb}{0.5,0.5,0.5}
\definecolor{mauve}{rgb}{0.58,0,0.82}

\lstset{frame=tb,
  language=Java,
  aboveskip=5mm,
  belowskip=3mm,
  showstringspaces=false,
  columns=flexible,
  basicstyle={\small\ttfamily},
  numbers=none,
  numberstyle=\tiny\color{gray},
  keywordstyle=\color{blue},
  commentstyle=\color{dkgreen},
  stringstyle=\color{mauve},
  breaklines=true,
  breakatwhitespace=true,
  tabsize=3
}

%----------------------------------------------------------------------------------------
%	TITLE SECTION
%----------------------------------------------------------------------------------------

\newcommand{\horrule}[1]{\rule{\linewidth}{#1}} % Create horizontal rule command with 1 argument of height

\title{	
\normalfont \normalsize 
\horrule{0.5pt} \\[0.4cm] % Thin top horizontal rule
\huge Week 02 \\ % The assignment title
\horrule{2pt} \\[0.5cm] % Thick bottom horizontal rule
}

\author{Seth Childers} % Your name

\date{} % Today's date or a custom date

\begin{document}

\maketitle % Print the title

%----------------------------------------------------------------------------------------
%	PROBLEM 1
%----------------------------------------------------------------------------------------

\section{Exercise 01}

\textbf{Problem:} For each of the following algorithms, indicate 
 (i) a natural size metric for its inputs, 
 (ii) its basic operation, and 
 (iii) whether the basic operation count can be different for inputs of the same size

\bigskip
\textbf{Code written in Markdown}
\begin{lstlisting}
# Exercises 2.1 - #01
## For each of the following algorithms, indicate 
 * (i) a natural size metric for its inputs, 
 * (ii) its basic operation, and 
 * (iii) whether the basic operation count can be different for inputs of the same size:

| Algorithm | Natural Size Metric of Input | Basic Operation | Basic Operation Count Can Be Different than Input Size |
| :--------------: | :---: | :--------------------------------: | :---: |
| a - computing the sum of *n* numbers | n | addition of two numbers | no |
| b - computing *n*! | n | multiplication of two numbers | no |
| c - finding the largest element in a list of *n* numbers | n | comparing the value of two elements | yes |
| d - Euclid's algorithm | n | Modulus | no |
| e - sieve of Eratosthenes | n | multiplication of two numbers | no |
| f - pen-and-pencil algorithm for multiplying two *n*-digit decimal integers | n | multiplication of two numbers | no |
\end{lstlisting}

\pagebreak
\textbf{Results}

\includegraphics[width=15cm, height=10cm]{./2-1_Exercise_01}
\pagebreak

%----------------------------------------------------------------------------------------
%	PROBLEM 2
%----------------------------------------------------------------------------------------

\section{Exercise 02}

\textbf{Problem:} For each of the following functions, indicate how much the function's value will change if its argument is increased fourfold.

\bigskip
\textbf{Code written in Markdown}
\begin{lstlisting}
# Exercises 2.1 - #08
## For each of the following functions, indicate how much the function`s value will change if its argument is increased fourfold.
 * log<sub>2</sub>*n*: log<sub>2</sub>4*n*/log<sub>2</sub>*n* = log<sub>2</sub>4*n* - log<sub>2</sub>*n* = (log<sub>2</sub>4 + log<sub>2</sub>*n*) - log<sub>2</sub>*n* = 2
 * sqrt(*n*): sqrt(4*n*)/sqrt(*n*) = 2*sqrt(*n*)/sqrt(*n*) = 2/1 = 2
 * *n*: (4*n*)/*n* = 4
 * *n*^2: (4*n)*^2/*n*^2 = (4*n*)^2/ (*n*^2) = 4^2 * *n*^2 = 16 * *n*^2 = 16
 * *n*^3: (4*n*)^3/*n*^3 = (4*n*)^3/ (*n*^3) = 4^3 * *n*^2* = 64 * *n*^2 = 64
 * 2^*n*: 2^4*n*/2^*n* = 2^3n
\end{lstlisting}

\textbf{Results}

\includegraphics[width=15cm, height=10cm]{./2-1_Exercise_08}
\pagebreak

%----------------------------------------------------------------------------------------
%	PROBLEM 3
%----------------------------------------------------------------------------------------

\section{Exercise 03}

\textbf{Problem:} For each of the following functions, indicate how much the function's value will change if its argument is increased fourfold.

\bigskip
\textbf{Code written in Markdown}
\begin{lstlisting}
# Exercise 2.2 - #03
## For each of the following functions, indicate the class Θ(g(*n*)) the function belongs to. (Use the simplest g(*n*) possible in your answers.) Prove your assertion.

|Function| Code |
| :----: | :---: |
| a | (n<sup>2</sup> + 1)<sup>10</sup> |
| b | sqrt(10*n*<sup>2</sup> + 7*n* + 3) |
| c | 2*n* lg(*n* + 2)<sup>2</sup> + (*n* + 2)<sup>2</sup> lg n/2 |
| d | 2<sup>*n* + 1</sup> + 3<sup>*n* - 1</sup> |
| e | ⌊ log<sub>2</sub>*n* ⌋ |
\end{lstlisting}

\textbf{Results}

\includegraphics[width=15cm, height=10cm]{./2-2_Exercise_03}
\includegraphics[width=15cm, height=10cm]{./UNADJUSTEDNONRAW_thumb_e9c}
\includegraphics[width=15cm, height=10cm]{./UNADJUSTEDNONRAW_thumb_e9d}
\includegraphics[width=15cm, height=10cm]{./UNADJUSTEDNONRAW_thumb_e9e}
\includegraphics[width=15cm, height=10cm]{./UNADJUSTEDNONRAW_thumb_e9f}
\includegraphics[width=15cm, height=10cm]{./UNADJUSTEDNONRAW_thumb_ea0}

\pagebreak

%----------------------------------------------------------------------------------------
%	PROBLEM 4
%----------------------------------------------------------------------------------------

\section{Exercise 04}

\textbf{Problem:} Compute the following sums.

(a.) 1 + 3 + 5 + 7 + ... + 999
(b.) 2 + 4 + 8 + 16 + ... + 1024
(c.) Summation (n + 1, i=3, 1)
(d.) Summation (n + 1, i=3, i)
(e.) Summation (n - 1, i=0, i + 1)
(f.) Summation (n, j=1, 3<sup>j + 1</sup>)
(g.) Summation (n, i=1 Summation (n, j=1, ij))
(h.) Summation (n, i=1, 1/i(i + 1))

\bigskip
\textbf{Code written in JavaScript}
\begin{lstlisting}
/**
* Exercises 2.3 - #01
* Compute the following sums.

* a. 1 + 3 + 5 + 7 + ... + 999
* b. 2 + 4 + 8 + 16 + ... + 1024
* c. Summation (n + 1, i=3, 1)
* d. Summation (n + 1, i=3, i)
* e. Summation (n - 1, i=0, i + 1)
* f. Summation (n, j=1, 3<sup>j + 1</sup>)
* g. Summation (n, i=1 Summation (n, j=1, ij))
* h. Summation (n, i=1, 1/i(i + 1))
*/
function a(currVal, max) {
    let total = 0;
    do {
        total += currVal;
        currVal += 2;
    } while (currVal <= max)
    return total;
}

function b(currVal, max) {
    let total = 0;
    do {
        total += currVal;
        currVal *= 2;
    } while (currVal <= max)
    return total;
}

function c(max) {
    let val = 0;
    for (let i=3; i <= max; i++) {
        val = i - 1;
    }
    return val;
}


console.log(`a. 1 + 3 + 5 + 7 + ... + 999 = ${a(1, 999)}`);
console.log(`b. 2 + 4 + 8 + 16 + ... + 1024 = ${b(2, 1024)}`);
console.log(`c. Summation (n + 1, i=3, 1) = ${c(8)}`);

\end{lstlisting}

\bigskip
\textbf{Results}
\begin{lstlisting}
a. 1 + 3 + 5 + 7 + ... + 999 = 250000
b. 2 + 4 + 8 + 16 + ... + 1024 = 2046
c. Summation (n + 1, i=3, 1) = 7
\end{lstlisting}

\includegraphics[width=15cm, height=10cm]{./UNADJUSTEDNONRAW_thumb_ea2}
\includegraphics[width=15cm, height=10cm]{./UNADJUSTEDNONRAW_thumb_ea3}
\includegraphics[width=15cm, height=10cm]{./UNADJUSTEDNONRAW_thumb_ea4}
\includegraphics[width=15cm, height=10cm]{./UNADJUSTEDNONRAW_thumb_eaf}
%----------------------------------------------------------------------------------------

\end{document}

\documentclass[12pt]{amsart}
\usepackage{geometry} % see geometry.pdf on how to lay out the page. There's lots.
\geometry{a4paper} % or letter or a5paper or ... etc
\usepackage[T1]{fontenc}
\usepackage[latin9]{inputenc}
\usepackage{amsmath}
\usepackage{amsaddr}
\usepackage{hyperref}
\usepackage{dirtytalk}
\usepackage{float}
\usepackage{listings}
\usepackage{hyperref}
\usepackage{enumerate}

\usepackage{color}
 
\definecolor{codegreen}{rgb}{0,0.6,0}
\definecolor{codegray}{rgb}{0.5,0.5,0.5}
\definecolor{stringcolor}{rgb}{0.7,0.23,0.36}
\definecolor{backcolour}{rgb}{0.95,0.95,0.92}
\definecolor{keycolor}{rgb}{0.007,0.01,1.0}
\definecolor{itemcolor}{rgb}{0.01,0.0,0.49}

\hypersetup{
    colorlinks=true,
    linkcolor=blue,
    filecolor=blue,      
    urlcolor=blue,
}
 
\lstdefinestyle{mystyle}{
    %backgroundcolor=\color{backcolour},   
    commentstyle=\color{codegreen},
    keywordstyle=\color{keycolor},
    numberstyle=\tiny\color{codegray},
    stringstyle=\color{stringcolor},
    basicstyle=\footnotesize,
    breakatwhitespace=false,         
    breaklines=true,                 
    captionpos=b,                    
    keepspaces=true,                 
    numbers=left,                    
    numbersep=5pt,                  
    showspaces=false,                
    showstringspaces=false,
    showtabs=false,                  
    tabsize=2
}
 
\lstset{style=mystyle}

\lstdefinelanguage{Swift}{
  keywords={associatedtype, class, deinit, enum, extension, func, import, init, inout, internal, let, operator, private, protocol, public, static, struct, subscript, typealias, var, break, case, continue, default, defer, do, else, fallthrough, for, guard, if, in, repeat, return, switch, where, while, as, catch, dynamicType, false, is, nil, rethrows, super, self, Self, throw, throws, true, try, associativity, convenience, dynamic, didSet, final, get, infix, indirect, lazy, left, mutating, none, nonmutating, optional, override, postfix, precedence, prefix, Protocol, required, right, set, Type, unowned, weak, willSet},
  ndkeywords={class, export, boolean, throw, implements, import, this},
  sensitive=false,
  comment=[l]{//},
  morecomment=[s]{/*}{*/},
  morestring=[b]',
  morestring=[b]"
}

\lstdefinelanguage{Erlang}{
  keywords={main, gcd },
  ndkeywords={rem, export, module},
  sensitive=false,
  comment=[l]{//},
  morecomment=[s]{/*}{*/},
  morestring=[b]',
  morestring=[b]"
}

\lstset{emph={Int,count,abs,repeating,Array}, emphstyle=\color{itemcolor}}


\title{About Algorithms}

\date{\today}

\lstset{style=mystyle}

%%% BEGIN DOCUMENT
\begin{document}
\maketitle
\section{Important Quotations}
\subsection{ What is an algorithm?}


  \say{An algorithm is a sequence of unambiguous instructions for solving a
  problem, i.e., for obtaining a required output from any legitimate
  input in a finite amount of time.}
  
  Or, to be precise, an algorithm is a \textbf{finite} sequence of
  unambiguous instructions, etc.


  --- Anany Levitin, page 3.
  \subsection{Algorithmic Thinking}
  \say{Computer Science and Engineering is a field that attracts a
  different kind of thinker. I believe that one who is a natural
  computer scientist thinks algorithmically. Such people are
  especially good at dealing with situations where different rules
  apply in different cases; they are individuals who can rapidly
  change levels of abstraction, simultaneously seeing things in-the-large and in-the-small.} --- Donald Knuth, as quoted in
  J. Hartmanis \textit{On computational complexity and the nature of computer
  science} published in \textit{Communications of the ACM 37}, 10
  (1994), 39.

  
  \subsection{More than Answers to Interview Questions}

  As one of the giants in the history of computer science put it:


 \say{ A person well-trained in computer science knows how to deal with
  algorithms: how to construct them, manipulate them, understand them,
  analyze them. This knowledge is preparation for much more than
  writing good computer programs; it is a general-purpose mental tool
  that will be a definite aid to the understanding of other subjects,
  whether they be chemistry, linguistic, or music, etc.  The reason
  for this may be understood in the following way:

  It has often been said that a person does not really understand
  something until after teaching it to someone else. Actually, a
  person does not \textbf{really} understand something until after teaching
  it to a \textbf{computer}, i.e., expressing it as an algorithm.}

  ---Donald Knuth, \textit{Computer Science and its relation to mathematics},
  The American Mathematical Monthly, 81(4), 1974


\subsection{ Why Study Algorithms?}

\say{Two ideas lie gleaming on the jeweler's velvet. The first is the calculus, the second, the algorithm.  The calculus and the rich body of mathematical analysis to which it gave rise made modern science possible; but it has been the algorithm that has made possible the modern world.}

  --- David Berlinski, \textit{The Advent of the Algorithm}, 2000
  
  \subsection{Compelling Reasons and Limitations}


  \say{There are compelling reasons to study algorithms. To put it bluntly,
  computer programs would not exist without algorithms. And with
  computer applications becoming indispensable in almost all aspects
  of our professional and personal lives, studying algorithms becomes
  a necessity for more and more people.

  Another reason for studying algorithms is their usefulness in
  developing analytical skills. After all, algorithms can be seen as
  special kinds of solutions to problems --- not answers but precisely
  defined procedures for getting answers. Consequently, specific
  algorithm design techniques can be interpreted as problem-solving
  strategies that can be useful regardless of whether a computer is
  involved.

  Of course, the precision inherently imposed by algorithmic thinking
  limits the kinds of problems that can be solved with an algorithm.

  You will not find, for example, an algorithm for living a happy life
  or becoming rich and famous.  On the other hand, this required
  precision has an important\ldots advantage.}

  --- Anany Levitin, \textit{Introduction to the Design and Analysis of
  Algorithms}, 2012
  \section{Basic, Commonly Known Algorithms}
  \subsection{ Algorithms for Computing Euclid's GCD}

  The Greatest Common Divisor (GCD) of two nonnegative, not-both-zero
  integers $m$ and $n$, denoted $gcd(m, n)$, is defined as the largest
  integer that divides both $m$ and $n$ evenly, i.e., with a remainder
  of zero.
  
  There are many algorithms that produce the gcd of two numbers. Not all are of the same quality. For the approaches shown below, $\exists m,n \in\mathbb{N}$.

\subsubsection{Approach 1}

   Apply repeatedly the equality
 
   $gcd(m, n) = gcd(n, m \pmod n)$ until $m \pmod n$ is equal to $0$.

   Since $gcd(m, 0) = m$, then $m$ is the greatest common divisor of
   $m$ and $n$.

   Example: $gcd(60, 24) = gcd(24, 12) = gcd(12, 0) = 12$.
   
Here is a mathematical representation of this algorithm
   
   \[ gcd(m,n) = \begin{cases}
    m\qquad\quad\quad\quad m\pmod n = 0\\
    gcd(m,n)\qquad otherwise 
\end{cases} 
\]
   
Here is this algorithm in Erlang.
\lstset{language=Erlang}
\lstinputlisting{gcd.erl}

   Now apply this algorithm to finding $gcd(31415, 14142)$.

   How do we know Euclid's algorithm will stop?

   Is each step non-ambiguous?

   Was the range of inputs specified?
   


\subsubsection{Approach 2}

\begin{enumerate}
   \item Assign the value of $min(m, n)$ to $t$
   \item Divide $m$ by $t$.  If the remainder of this division is $0$, go
      to step 3; otherwise go to step 4.
   \item Divide $n$ by $t$.  If the remainder of this division is $0$,
      return the value of $t$ as the answer and stop; otherwise
      proceed to step 4
   \item Decrease the value of $t$ by $1$. Go to Step 2.
\end{enumerate}
   
   Apply this algorithm to finding $gcd(31415, 14142)$.

   What is wrong with this algorithm?

\subsubsection{Approach 3}

   1. Find the prime factors of $m$.
   2. Find the prime factors of $n$.
   3. Identify all common factors in the two prime expansions found in
      steps 1 and 2.
   4. Compute the product of all the common factors and return it as
      the greatest common divisor of $m$ and $n$.

   Example: Calculate $gcd(60, 24)$
   \begin{align}
   60 &= 2 \cdot 2 \cdot 3 \cdot 5\\
   24 &= 2 \cdot 2 \cdot 2 \cdot 3\\
   gcd(60, 24) &= 2 \cdot 2 \cdot 3 = 12
   \end{align}

   Apply this algorithm to finding $gcd(31415, 14142)$.

   Problems with this algorithm?




\end{document}